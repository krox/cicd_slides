\documentclass[english,xcolor=pst,11pt]{beamer}

%\usetheme{Rochester}
%\usetheme{Berkeley}
\usetheme{Berlin}
\usepackage{beamerthemesplit}


\usepackage[utf8]{inputenc}
\usepackage{amsmath, amssymb, bbm}
\usepackage{url}
\usepackage{hyperref}
\usepackage{physics, slashed}
\usepackage{graphicx}
\usepackage{hyperref}
\usepackage{placeins}
\usepackage{calc}
\usepackage{float}
\usepackage{xcolor}
%\usepackage{accents}
\usepackage{stackrel}
\usepackage{mathtools}
\usepackage{graphicx}
\usepackage{adjustbox}

\usepackage{tikz}
\usetikzlibrary{patterns,math,calc}
\usetikzlibrary{decorations.pathreplacing}
\usetikzlibrary{positioning}
\usetikzlibrary{decorations.pathmorphing}
\usetikzlibrary{decorations.markings}
\usetikzlibrary{arrows}
\renewcommand\epsilon\varepsilon % its probably not consistent in the tex code
\renewcommand\nabla\partial % its probably not consistent in the tex code



\title[Langevin Integration and Correlated Markov Chains]{Automatic testing of Lattice QCD Software on Tursa}
\author{Simon Bürger}
\date{July 25th, 2024}

% \pdfinfo{
%   /Title    (Scattering in Lattice Systems)
%   /Author   (Simon Bürger)
% }

\newlength\leftsidebar
\newlength\rightsidebar
\makeatletter
\setlength\leftsidebar{\beamer@leftsidebar}
\setlength\rightsidebar{\beamer@rightsidebar}
\makeatother

\begin{document}

\maketitle

\begin{frame}
 \begin{block}{Overview}
  \begin{enumerate}
   \item Lattice QCD basics
   \item Integration schemes for the Lanvgevin Equation
   \item Numerical results
   \item Correlated Markov chains
  \end{enumerate}
 \end{block}

\end{frame}



\section{Lattice QCD}

% Steps:
%   - continuum action
%   - Wick rotation -> probability distribution
%   - pseudo-fermions -> matrix inversion
%   - discretization -> going from gauge algebra to gauge group
%   - doubling -> extra wilson term
\begin{frame}

% \begin{align*}
%   L[\psi,\overline\psi,A] &= \sum_f\overline{\psi}_f(i\slashed{D}-m_f)\psi_f -  \frac{1}{2g^2} \trace F_{\mu\nu}F^{\mu\nu}\,, \\
%   D_\mu &= \partial_\mu - iA_\mu\,, \\
%   F_{\mu\nu} &= i[D_\mu,D_\nu]
% \end{align*}


Continuum Lagrangian (Euclidean metric)
\begin{align*}
 L[\psi,\overline\psi,A] &= \sum_f\overline\psi_f (\slashed D + m_f)\psi_f + \frac{1}{2g^2}\trace F_{\mu\nu}F_{\mu\nu} \\
 F_{\mu\nu} &= -i[D_\mu,D_\nu] \\
 D_\mu &= \partial_\mu + iA_\mu\
\end{align*}

Observable expecation values:
\begin{align*}
 \langle \mathcal O \rangle &= \frac{1}{Z} \int D\psi\, D\overline\psi\, DA\,\mathcal O(\psi,\overline\psi,A)\, e^{-S} \\
 S &= \int_{\mathbb{R}^4}dx\, L
\end{align*}
\end{frame}



\end{document}
